\documentclass[12pt]{article}
\usepackage{epsfig}
\usepackage{amsmath}
\usepackage{mathtools}
\usepackage{times}
\renewcommand{\topfraction}{1.0}
\renewcommand{\bottomfraction}{1.0}
\renewcommand{\textfraction}{0.0}
\setlength {\textwidth}{6.6in}
\hoffset=-1.0in
\oddsidemargin=1.00in
\marginparsep=0.0in
\marginparwidth=0.0in                                                                               
\setlength {\textheight}{9.0in}
\voffset=-1.00in
\topmargin=1.0in
\headheight=0.0in
\headsep=0.00in
\footskip=0.50in                                         
\setcounter{page}{1}
\begin{document}
\def\pos{\medskip\quad}
\def\subpos{\smallskip \qquad}
\newfont{\nice}{cmr12 scaled 1250}
\newfont{\name}{cmr12 scaled 1080}
\newfont{\swell}{cmbx12 scaled 800}

\begin{center}
    \textbf{PHYS 20323/60323: Fall 2020 - LaTeX Example}
\end{center}

\begin{enumerate}
\item Consider a particle confined in a two-dimensional infinite square well
\[ V(x,y) =
  \begin{cases*}
    $0$, & if $0 \leq x \leq a$ and $0 \leq y \leq a$ \\
    $\infty$, & otherwise
  \end{cases*}\]
 The eigenfunctions have the form:
 \begin{equation*}
     \Psi(x,y) = \frac{2}{a}\sin{\frac{n\pi x}{a}}\sin{\frac{m\pi y}{a}}
 \end{equation*}
with the corresponding energies being given by:
\begin{equation*}
    E_{nm} = (n^2 + m^2) \frac{\pi^2\hbar^2}{2ma^2}
\end{equation*}
\begin{enumerate}
    \item (5 points) What are the levels of degeneracy of the five lowest energy values? \newline
    \item (5 points) Consider a perturbation given by:
    \begin{equation*}
        \hat{H'} = a^2V_0\delta\left(x-\frac{a}{2}\right)\delta\left(y-\frac{a}{2}\right)
    \end{equation*}
    Calculate the first order correction to the ground state energy. \newline
\end{enumerate}
\item \textbf{The following questions refer to stars in the Table below.} \newline
Note: There may be multiple answers \newline \newline
\begin{tabular}{|c|c|c|c|c|c|c|}
\hline
   Name & Mass & Luminosity & Lifetime & Temperature & Radius \\
   \hline
    Zeta  & $60.M_{sun}$ & $10^6 L_{sun}$  & $8.0 \times 10^5$ years & &  \\
    \hline
    Epsilon & $6.0 M_{sun}$ & $10^3 L_{sun}$ & & $20,000$ K & \\
    \hline
    Delta & $2.0M_{sun}$ & & $5.0 \times 10^8$ years & & $2R_{sun}$\\
    \hline
    Beta & $1.3M_{sun}$ & $3.5L_{sun}$ & & & \\
    \hline 
    Alpha & $1.0M_{sun}$ & & & & 1 $R_{sun}$ \\ 
    \hline 
    Gamma & $0.7M_{sun}$ & & $4.5 \times 10^{10}$ years & $5000$ K & \\
    \hline 
\end{tabular}
\begin{enumerate}
    \item (4 points) Which of these stars will produce a planetary nebula at the end of their life. \newline
    \item (4 points) Elements heavier than Carbon will be produced in which stars.
\end{enumerate}
\end{enumerate}
\end{document}